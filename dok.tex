\documentclass[11pt, a4paper]{article}
\usepackage[T1]{fontenc}
\usepackage{polski}
\usepackage[utf8]{inputenc}
\usepackage{listings}
\usepackage{lmodern}
\usepackage{graphicx}
\usepackage{tabularx}
\usepackage{hyperref}


\title{Ray casting - dokumentacja}
\author{Mikołaj Kubik 291083}

\lstset{
    language=C++,
    keywordstyle=\color{blue},
    stringstyle=\color{red},
    commentstyle=\color{green},
    morecomment=[l][\color{magenta}]{\#}
}

\lstset{
    breaklines=true,
    framexrightmargin=0em,
    framexleftmargin=0em,
}

\hypersetup{
    colorlinks=true,
    linkcolor=black,
    filecolor=magenta,      
    urlcolor=black,
}
 
\urlstyle{same}


\begin{document}
\maketitle
\tableofcontents
\newpage

\section{Wstęp}

\subsection{Ray casting}

\subsubsection{Definicja}
Ray casting to technika renderowania trójwymiarowych scen na podstawie dwuwymiarowej mapy. Generowanie perspektywy 3D odbywa się za pomocą algorytmu:
\begin{itemize}
    \item generowanie promienia wycinka koła pola widzenia kamery,
    \item znalezienie końca promienia (punktu przecięcia z obiektem na mapie, końca mapy lub punktu odległego od kamery o długość promienia),
    \item wygenerowanie fragmentu widoku na podstawie odległości między kamerą a końcem promienia.
\end{itemize}

\subsubsection{Schemat}
\includegraphics[width=\textwidth]{example.png}

\subsection{Działanie programu}
Zadaniem programu jest prezentacja metody raycastingu. Losowo generuje on macierz rzadko wypełnioną elementami o 
z góry zadanej ilości wartości, odpowiadającymi obiektom. W praktyce, z punktu widzenia użytkownika program dla każdej różnej od zera 
wartości macierzy generuje prostopadłościan o takiej wysokości aby stworzyć złudzenie perspektywy i w kolorze odpowiadającym tej wartości.
\newline
\newline
Zrzut ekranu z wywołania programu:
\newline
\includegraphics[width=\textwidth]{program.png}

\section{Użyte biblioteki}
\subsection{SFML}
\underline{\href{www.sfml-dev.org}{SFML}}(Simple and Fast Multimedia Library) to biblioteka zapewniająca interfejs do obsługi urządzeń multimedialnych. W programie
wykorzystywana jest do tworzenia okna, wyświetlania obrazu oraz obsługi zdarzeń związanych z klawiaturą i myszą.
\subsection{Google Test}
\underline{\href{https://github.com/google/googletest}{Google Test}} to otwarta biblioteka pozwalająca na testowanie oprogramowania pisanego 
w językach C i C++.

\section{Uruchomienie}
Za kompilację, testowanie i czyszczenie odpowiada skrypt MakeFile, przechowujący zależności między poszczególnymi plikami.
\subsection{Program}
Aby utworzyć plik wykonywalny programu należy użyć polecenia
\begin{lstlisting}
~ RayCasting git:(master) make
\end{lstlisting}
W odpowiedzi powinny pojawić się polecenia wykonywane aby ostatecznie utworzyć plik wykonywalny \textit{raycasting}.
\begin{lstlisting}
g++ main.cpp -c -o main.o
g++ Map.cpp -c -o Map.o
g++ View.cpp -c -o View.o
g++ Actor.cpp -c -o Actor.o
g++ Raycaster.cpp -c -o Raycaster.o
g++ main.o Map.o View.o Actor.o Raycaster.o -lsfml-graphics -lsfml-window -lsfml-system -g -o raycasting
\end{lstlisting}
Program może teraz zostać uruchomiony. Aby wyłączyć program należy użyć klawisza \textbf{ESC}.
\subsection{Testy}
Aby uruchomić testowanie należy użyć polecenia make z parametrem test. Odpowiednie pliki zostaną skompilowane oraz uruchomiony 
zostanie nowo utworzony plik wykonywalny \textit{test}.
\begin{lstlisting}
~ RayCasting git:(master) make test

g++ tests.cpp -c -o tests.o
g++ tests.o Map.o View.o Actor.o Raycaster.o 
-lsfml-graphics -lsfml-window 
-lsfml-system -pthread -lgtest -g -o test
./test

...

[----------] Global test environment tear-down
[==========] 9 tests from 3 test cases ran.
[  PASSED  ] 9 tests.  
\end{lstlisting}
Po wykonaniu polecenia w oknie konsoli powinny wyświetlić się wyniki testów jednostkowych.
\subsection{Czyszczenie}
Pliki wygenerowane podczas kompilacji programu możemy usunąć za pomocą polecenia make z parametrem clean. Uruchomi ono skrypt 
z pliku makefile usuwający wszystie pliki z rozszerzeniem .o oraz pliki wykonywalne raycasting i test.
\begin{lstlisting}
~ RayCasting git:(master) make clean 

rm -f *.o raycasting test
\end{lstlisting}

\section{Implementacja}
Program składa się z czterech podstawowych klas oraz dwóch plików zawierających funkcję \textit{main()}.
\subsection{Omówienie poszczególnych klas}
\subsubsection{Map}
Klasa zawierająca informacje o przestrzeni, która ma zostać wygenerowana w postaci widoku 3D i po której może poruszać się kamera.
\paragraph{Pola}: 
\begin{lstlisting}
    int m_width;
    int m_height;
    int **m_map;
\end{lstlisting}
Klasa Map zawiera macierz, która zostaje wypełniona zerami oraz wymiary tej macierzy.
\paragraph{Metody}: 
Głównymi metodami klasy Map są metody dostępowe, pilnujące aby odwoływać się jedynie do prawidłowych elementów macierzy:
\begin{lstlisting}
int Map::get(int x, int y)
{
    if (x >= 0 && x < m_width && y >= 0 && y < m_height)
    {
        return m_map[x][y];
    }

    return -1;
}

void Map::set(int x, int y, int value)
{
    if (x >= 0 && x < m_width && y >= 0 && y < m_height)
    {
        m_map[x][y] = value;
    }
}
\end{lstlisting}

Klasa posiada także metodę
\begin{lstlisting}
void Map::processEvents(std::list<point_t> *points)
{
    point_t point;
    while (points->size() > 0)
    {
        point = points->front();
        points->pop_front();
    }
}
\end{lstlisting}
która otrzymuje listę punktów, w których miały miejsce zdarzenia myszy aby umożliwić implementację interakcji z otoczeniem. 

\subsubsection{Actor}
Klasa zawierająca informacje o aktorze/kamerze, która porusza się po wygenerowanym widoku.
\paragraph{Enum}:
\begin{lstlisting}
enum direction
{
    LEFT,
    RIGHT,
    UP,
    DOWN
};
\end{lstlisting}
jest wykorzystywany w celu przekazania informacji w którym kierunku ma przemieścić się aktor.
\paragraph{Pola}: 
\begin{lstlisting}
public:

double m_x{0}; // pozycja na mapie, wzpolzedna X
double m_y{0}; // pozycja na mapie wspolzedna Y

private:

int m_view_angle{90}; // pole widzenia
double m_heading{0}; // kierunek w ktory zwrocona jest kamera
double m_rotation_speed{1}; // predkosc obrotu kamery
double m_movement_speed{1}; // predkosc przemieszczania sie
Map *m_map; // mapa, po ktorej aktor sie porusza
\end{lstlisting}
Współrzędne kamery opisywane są za pomocą zmiennych typu double aby zapewnić płynność ruchu przy niskiej rozdzielczości mapy.
\paragraph{Metody}: 
Metoda move:
\begin{lstlisting}
void Actor::move(direction direction)
{
    double x = m_x;
    double y = m_y;
    switch (direction)
    {
    case UP:
            x = m_x + m_movement_speed * cos(m_heading * M_PI / 180);
            y = m_y + m_movement_speed * sin(m_heading * M_PI / 180);
        break;

    case DOWN:
            x = m_x + m_movement_speed * cos((m_heading + 180) * M_PI / 180);
            y = m_y + m_movement_speed * sin((m_heading + 180) * M_PI / 180);   
        break;

    case LEFT:
            x = m_x + m_movement_speed * cos((m_heading - 90) * M_PI / 180);
            y = m_y + m_movement_speed * sin((m_heading - 90) * M_PI / 180);
        break;

    case RIGHT:
            x = m_x + m_movement_speed * cos((m_heading + 90) * M_PI / 180);
            y = m_y + m_movement_speed * sin((m_heading + 90) * M_PI / 180);
        break;

    default:
        break;
    }

    if (x > 0 && x < m_map->getWidth() && y > 0 && y < m_map->getHeight() && m_map->get((int)x, (int)y) == 0)
    {
        m_x = x;
        m_y = y;
    }
}
\end{lstlisting}
odpowiada za zmianę położenia aktora w zależności od kierunku, w który jest zwrócony oraz za zatrzymanie ruchu w momencie, gdy napotkana 
zostanie przeszkoda.
\newline
Metoda rotate:
\begin{lstlisting}
void Actor::rotate(direction dir)
{
    switch (dir)
    {
    case LEFT:
        m_heading - m_rotation_speed >= 0 ? m_heading -= m_rotation_speed : m_heading = 360 - m_rotation_speed;
        break;

    case RIGHT:
        m_heading + m_rotation_speed <= 360 ? m_heading += m_rotation_speed : m_heading = 0 + m_rotation_speed;
        break;
    }
}
\end{lstlisting}
odpowiada za obrót kamery w odpowiednim kierunku. Kierunek kamery wyrażony jest w stopniach w przedziale od 0 do 360.
\newline
Metoda processEvents:
\begin{lstlisting}
void Actor::processEvents(std::list<int> *keys)
{
    while (keys->size() > 0)
    {
        switch (keys->front())
        {
        case 0:
            rotate(LEFT);
            break;

        case 3:
            rotate(RIGHT);
            break;

        case 71:
            move(LEFT);
            break;
        case 72:
            move(RIGHT);
            break;

        case 73:
            move(UP);
            break;

        case 74:
            move(DOWN);
            break;

        default:
            break;
        }
        keys->pop_front();
    }
}
\end{lstlisting}
jako argument przyjmuje listę wartości oddpowiadających naciśniętym przyciskom. W tej metodzie zdefiniowane są zależności 
pomiędzy wciśniętym przyciskiem a akcją, jaka ma zostać wykonana.
\subsubsection{View}
Klasa View jest wyodrębnioną klasą odpowiedzialną za wyświetlanie obrazu oraz rejestrowanie interakcji z użytkownikiem.
\paragraph{Pola}: 
\begin{lstlisting}
int m_window_width{800};
int m_window_height{600};
std::vector<sf::Color> m_colors;
\end{lstlisting}
Klasa zawiera informacje o wielkości okna, jakie ma wyświetlać oraz o kolorach, które mogą posiadać wyświetlane obiekty.
\paragraph{Metody}: 
Klasa view jest zaimplementowana z wykorzystaniem biblioteki SFML. Część z metod bezpośrednio odpowiada metodom sfml
jednak zostały zamknięte w dodatkowych aby ułatwić zmianę narzędzia do generacji obrazu.
\newline
Metoda checkEvents:
\begin{lstlisting}
void View::checkEvents(std::list<int> *keys, std::list<point_t> *points)
{
    sf::Event event;
    while (window.pollEvent(event))
    {
        switch (event.type)
        {
        case sf::Event::Closed:
            window.close();
            break;

        case sf::Event::KeyPressed:
            if (event.key.code == sf::Keyboard::Escape)
            {
                window.close();
            }
            break;

        case sf::Event::MouseButtonPressed:
            point_t clicked;
            clicked.x = event.mouseButton.x;
            clicked.y = event.mouseButton.y;
            points->push_back(clicked);
            break;

        default:
            break;
        }
    }

    if(sf::Keyboard::isKeyPressed(sf::Keyboard::Up)){
        keys->push_back(sf::Keyboard::Up);
    }

    if(sf::Keyboard::isKeyPressed(sf::Keyboard::Down)){
        keys->push_back(sf::Keyboard::Down);
    }

    if(sf::Keyboard::isKeyPressed(sf::Keyboard::Left)){
        keys->push_back(sf::Keyboard::Left);
    }

    if(sf::Keyboard::isKeyPressed(sf::Keyboard::Right)){
        keys->push_back(sf::Keyboard::Right);
    }

    if(sf::Keyboard::isKeyPressed(sf::Keyboard::A)){
        keys->push_back(sf::Keyboard::A);
    }

    if(sf::Keyboard::isKeyPressed(sf::Keyboard::D)){
        keys->push_back(sf::Keyboard::D);
    }

}
\end{lstlisting}
odpowiada za wstępne przetworzenie (wyjście z aplikacji) i eksportowanie zdarzeń związanych z klawiaturą i myszą. Biblioteka 
sfml gromadzi informacje o zdarzeniach a metoda zajmuje się ich interpretacją na potrzeby programu. Zdarzenia związane z przemieszczaniem aktora 
są eksportowane na podstawie informacji czy klawisz jest wciśnięty natomiast pozostałe zdarzenia klawiaturowe reagują jedynie na naciśnięcie. 
Rozwiązanie to pozwala na płynniejsze i bardziej intuicyjne sterowanie. 
\newline
Metoda paint line:
\begin{lstlisting}
void View::paint_line(double x1, double y1, double x2, double y2, int color_index)
{
    double dist = y2 - y1;
    double mul = 1;

    if(dist < 10){
        mul = 0.5;
    }
    else{
        mul = dist/100 + 0.5;
    }

    mul > 1 ? mul = 1 : true;

    sf::Color color;

    if(color_index == -1){
        color = sf::Color(255, 255, 255, 255);
    }
    else if(color_index > m_colors.size()){
        color = sf::Color(0, 0, 0, 0);
    }
    else{
        color = m_colors[color_index - 1];
    }

    sf::VertexArray line(sf::Lines, 2);
    line[0].position = sf::Vector2f(x1, y1);
    line[0].color = sf::Color(color.r*mul, color.g*mul, color.b*mul, 255);
    line[1].position = sf::Vector2f(x2, y2);
    line[1].color = sf::Color(color.r*mul, color.g*mul, color.b*mul, 255);

    window.draw(line);
}    
\end{lstlisting}
służy do rysowania odcinka między zadanymi punktami i o zadanym kolorze. Dodatkowo, ponieważ algorytm ray castingu wymaga rysowania jedynie
pionowych odcinków, odległość pomiędzy współrzędnymi y obu punktów wykorzystywana jest do określenia jak blisko kamery znajduje się dany odcinek.
Im dalej od kamery tym ciemniejszy jest generowany odcinek, aby zasymulować problemy z dostrzeganiem szczegółów oddalonych obiektów. 
\newline
Metoda paint pixel:
\begin{lstlisting}
void View::paint_pixel(double x, double y, int color_index){
    sf::Color color;
    if(color_index > m_colors.size()){
        color = sf::Color(0, 0, 0, 0);
    }
    else{
        color = m_colors[color_index - 1];
    }

    sf::VertexArray point(sf::Points, 1);
    point[0].position = sf::Vector2f(x, y);
    point[0].color = color;

    window.draw(point);
}   
\end{lstlisting}
pozwala na pokolorowanie pojedynczego pixela na ekranie, dzięki czemu może zostać wykorzystana do rozbudowy interfejsu użytkownika.

\subsubsection{Raycaster}
\paragraph{Pola}: 
\paragraph{Metody}: 

\subsection{Omówienie głównego pliku programu - main.cpp}

\section{Testy}


\end{document}